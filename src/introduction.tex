\chapter{Introduction}
\label{ch:introduction}

Rust was first released in 2015, and has since then amazed programmers all around the globe~\cite{RustWikipedia}. It is both a systems language and its compiler can guarantee memory safety. These two features rarely come together. An investigation performed by Microsoft showed that about 70\% of the bugs they found in their codebase, are related to memory safety issues~\cite{MemeorySafety70}. These could be avoided completely if a memory safe language were used. Furthermore, the Rust development is based on the nearly 40 years of experience with C and other related languages. Therefore, Rust comes out of the box with the most modern features and with the convenience of a recently developed language, like built-in support for hash tables.

These are some of the reasons why Microsoft, Google and Linux started using Rust in their codebases \cite{MicrosoftRust},\cite{GoogleRust},\cite{LinuxRust}. Consequently, Rust could also be an opportunity for the \gls{pulp} platform. Therefore, we decided to give Rust a chance.

In the embedded setup, a \gls{rtos} is needed for most operations. Since Rust is much younger than its competitors C and C++, there do not exist many embedded operating systems yet. And the few that exist, are mostly tailor-made for the ARM Cortex-M architecture. Therefore, the available \gls{rtos}s are evaluated in section \ref{sec:rtos_evaluation}.

The most fitting \gls{rtos}, \gls{rtic}, was ported to the RISC-V architecture, see section \ref{sec:rtic_port}. Since \gls{rtic} is heavily reliant on interrupts, a platform with a powerful interrupt controller was chosen. The choice fell on the ControlPULP IP, that features the \acrfull{clic}, that is closely related to the \gls{nvic}, the interrupt controller of ARM Cortex-M.

Since the \gls{clic} appeared only very recently, there was no support for it in any Rust libraries. Support had to be added by us, see section \ref{sec:libraries}.

The porting of the RTOS has two main benefits. First, it allows Rust programs that need an operating system to run on the \gls{pulp} platform. Secondly, it allows us to compare its performance on ARM and RISC-V, and therefore to identify possible drawbacks of the hardware implementation of the ControlPULP IP, or of the newly written \gls{clic} libraries.

In the first iteration, the performance of \gls{rtic} was much better on ARM than on RISC-V. See section \ref{sec:results_original_implementation}. By utilizing the \texttt{mnxti} \gls{csr} of the \gls{clic} we were able to improve the performance of \gls{rtic} on RISC-V by a significant amount. See section \ref{sec:results_nxti}. Furthermore, we could estimate with high confidence, that the usage of the \texttt{fastirq} hardware extension for the \gls{clic} would bring the performance on both architectures to a very comparable level. See section \ref{sec:results_fastirq} for the final results.
