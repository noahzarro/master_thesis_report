\chapter{Conclusion and Future Work}
\label{ch:conclusion}
\label{sec:future_work}

The evaluation shows that the performance of \gls{rtic} on ARM and RISC-V is comparable.
Even though, the first RISC-V implementation's performance was far behind the performance of the ARM implementation, it could be improved significantly by the usage of the \texttt{nxti} \gls{csr}. Furthermore, It can be estimated with high confidence, that the usage of the \texttt{fastirq} hardware extension would increase performance to a level that is very close to ARM.

Naturally, the actual inclusion of the \texttt{fastirq} extension into the test setup would be the next step to perform. Another interesting project would be to use the RISC-V E-Extension and EABI, that could possibly improve the interrupt handling performance even further.

In the process, we created and updated several Rust crates that can be used in the RISC-V embedded setup. These crates can now be used inside the research group, but since they are also published under an open source license, they can be used by the whole Rust community. This could also accelerate the development of a Rust RISC-V embedded community, which is desirable for further integration of Rust into the \gls{pulp} platform.

Furthermore, the insights gained while porting \gls{rtic} to RISC-V will also benefit the developers of \gls{rtic} that are currently working on a second version of \gls{rtic} that offers native RISC-V support.

Lastly, a foundation was laid for further Rust development on the PULP platform. For potential next Rust projects, there now exists a runtime with interrupt support and an architecture crate that offers access to all \gls{clic} functionalities. And most importantly, now there exists a reference project with examples that ease the starting of a new project.


