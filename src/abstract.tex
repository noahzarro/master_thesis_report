\chapter*{Abstract}

In the last few years, the Rust programming language strongly grew in popularity. And since it can be used as a memory save systems language, it is increasingly used in embedded programming. However, existing embedded libraries strongly focus on the ARM architecture.
The goal of this thesis was to evaluate and port an existing \gls{rtos} to the ControlPULP IP, that implements a RISC-V core. The choice fell on RTIC, a well maintained lightweight \gls{rtos}.

In the process, we created or extended several Rust RISC-V libraries to add support for interrupt handling, especially with the \gls{clic}.

With the \gls{rtos} ported to both platforms, ARM and RISC-V, it was possible to perform performance measurements on both architectures go gain insights on where the ControlPULP IP would have potential for improvements. 

The measurements mostly focused on the latency of task spawns, since these are the most important core transitions in RTIC. They were performed by reading out the cycle count registers that are provided on both platforms.

With several improvements to the ControlPULP IP applied, the RTOS reaches comparable performance to the original ARM implementation. However, there is still room for further improvements and analysis.